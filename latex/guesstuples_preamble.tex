\usepackage[T1]{fontenc}
\usepackage[utf8]{inputenc}
%\usepackage{mathptmx}  % {mathptmx}  % {tgtermes} 

\usepackage{newtxtext}
\usepackage{newtxmath}

\usepackage[dayofweek]{datetime}
\newdateformat{UKvardate}{\THEDAY\ \monthname[\THEMONTH] \THEYEAR}
\UKvardate

% Narrower margins from https://kb.mit.edu/confluence/pages/viewpage.action?pageId=3907057#HowcanIchangethemarginsinLaTeX%3F-Changingmarginsbyhand

\addtolength{\oddsidemargin}{-.775in}
\addtolength{\evensidemargin}{-.775in}
\addtolength{\textwidth}{1.55in}

\addtolength{\topmargin}{-.875in}
\addtolength{\textheight}{1.75in}


\usepackage{hyperref}
\usepackage{siunitx}
\sisetup{detect-all}



\newcommand{\sref}[1]{Section~\ref{#1}}
\newcommand{\fref}[1]{Figure~\ref{#1}}

% From https://tug.org/FontCatalogue/beramono/
\usepackage[scaled=0.85]{beramono}
\usepackage[T1]{fontenc}


% From https://stackoverflow.com/questions/3175105/inserting-code-in-this-latex-document-with-indentation
\usepackage{listings}
\usepackage{xcolor}
\usepackage{hyperref}
%\usepackage{amsmath, amssymb} % not needed (and must be omitted) with newtxmath

\definecolor{offwhite}{HTML}{F8F8F2}
\definecolor{nearblack}{HTML}{272822}
\definecolor{stringyellow}{HTML}{E6DB74}
\definecolor{keywordcyan}{HTML}{66D9EF}

\lstset{frame=tb,
	language=Python,
	backgroundcolor=\color{nearblack},
	aboveskip=3mm,
	belowskip=3mm,
	showstringspaces=false,
	xleftmargin=6pt,
	xrightmargin=6pt,
	frame=single,
	framesep=6pt,
	framerule=0pt,
	columns=flexible,
	basicstyle={\ttfamily\color{offwhite}},
	numbers=none,
	numberstyle=\tiny\color{offwhite},
	keywordstyle=\color{keywordcyan},
	commentstyle=\color{lightgray},
	stringstyle=\color{stringyellow},
	breaklines=true,
	breakatwhitespace=true,
	tabsize=4
}

\usepackage{graphicx}

\newcommand{\entry}[1]{\subsection*{#1}}
\newcommand{\rentry}[1]{#1}

\newcommand{\eref}[1]{Eq.~\ref{#1}}
\newcommand{\rcite}[1]{Ref.~\cite{#1}}

\renewcommand{\vec}[1]{\boldsymbol{#1}}
\newcommand{\mat}[1]{\mathbf{#1}}
\newcommand{\abs}[1]{\left\lvert #1 \right\rvert}
\newcommand{\x}{\vec{x}}
\newcommand{\y}{\vec{y}}
\newcommand{\z}{\vec{z}}
\renewcommand{\c}{\vec{c}}


% https://tex.stackexchange.com/questions/42619/x-mark-to-match-checkmark
\usepackage{pifont}
\newcommand{\cmark}{\ding{51}}%
\newcommand{\xmark}{\ding{55}}
\newcommand{\rt}[1]{r_{#1\text{\tiny \cmark}}}
\newcommand{\rf}[1]{r_{#1\text{\tiny \xmark}}}

%\usepackage{natbib}