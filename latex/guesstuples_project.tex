\documentclass[12pt]{article}

\usepackage[T1]{fontenc}
\usepackage[utf8]{inputenc}
%\usepackage{mathptmx}  % {mathptmx}  % {tgtermes} 

\usepackage{newtxtext}
\usepackage{newtxmath}

\usepackage[dayofweek]{datetime}
\newdateformat{UKvardate}{\THEDAY\ \monthname[\THEMONTH] \THEYEAR}
\UKvardate

% Narrower margins from https://kb.mit.edu/confluence/pages/viewpage.action?pageId=3907057#HowcanIchangethemarginsinLaTeX%3F-Changingmarginsbyhand

\addtolength{\oddsidemargin}{-.775in}
\addtolength{\evensidemargin}{-.775in}
\addtolength{\textwidth}{1.55in}

\addtolength{\topmargin}{-.875in}
\addtolength{\textheight}{1.75in}


\usepackage{hyperref}
\usepackage{siunitx}
\sisetup{detect-all}



\newcommand{\sref}[1]{Section~\ref{#1}}
\newcommand{\fref}[1]{Figure~\ref{#1}}

% From https://tug.org/FontCatalogue/beramono/
\usepackage[scaled=0.85]{beramono}
\usepackage[T1]{fontenc}


% From https://stackoverflow.com/questions/3175105/inserting-code-in-this-latex-document-with-indentation
\usepackage{listings}
\usepackage{xcolor}
\usepackage{hyperref}
%\usepackage{amsmath, amssymb} % not needed (and must be omitted) with newtxmath

\definecolor{offwhite}{HTML}{F8F8F2}
\definecolor{nearblack}{HTML}{272822}
\definecolor{stringyellow}{HTML}{E6DB74}
\definecolor{keywordcyan}{HTML}{66D9EF}

\lstset{frame=tb,
	language=Python,
	backgroundcolor=\color{nearblack},
	aboveskip=3mm,
	belowskip=3mm,
	showstringspaces=false,
	xleftmargin=6pt,
	xrightmargin=6pt,
	frame=single,
	framesep=6pt,
	framerule=0pt,
	columns=flexible,
	basicstyle={\ttfamily\color{offwhite}},
	numbers=none,
	numberstyle=\tiny\color{offwhite},
	keywordstyle=\color{keywordcyan},
	commentstyle=\color{lightgray},
	stringstyle=\color{stringyellow},
	breaklines=true,
	breakatwhitespace=true,
	tabsize=4
}

\usepackage{graphicx}

\newcommand{\entry}[1]{\subsection*{#1}}
\newcommand{\rentry}[1]{#1}

\newcommand{\eref}[1]{Eq.~\ref{#1}}
\newcommand{\rcite}[1]{Ref.~\cite{#1}}

\renewcommand{\vec}[1]{\boldsymbol{#1}}
\newcommand{\mat}[1]{\mathbf{#1}}
\newcommand{\abs}[1]{\left\lvert #1 \right\rvert}
\newcommand{\x}{\vec{x}}
\newcommand{\y}{\vec{y}}
\newcommand{\z}{\vec{z}}
\renewcommand{\c}{\vec{c}}


% https://tex.stackexchange.com/questions/42619/x-mark-to-match-checkmark
\usepackage{pifont}
\newcommand{\cmark}{\ding{51}}%
\newcommand{\xmark}{\ding{55}}
\newcommand{\rt}[1]{r_{#1\text{\tiny \cmark}}}
\newcommand{\rf}[1]{r_{#1\text{\tiny \xmark}}}

\newcommand{\figstamp}[2]{\texttt{#1\_#2}}

%\usepackage{natbib}

%opening
\title{GuessTuples Project}
\author{Andrew J. Wren}

\begin{document}

\maketitle

\begin{abstract}
	Notes on GuessTuples project
\end{abstract}


\section{Configuring the nets}

\subsection{Alice}

The input array to guess is $\vec{x}.$  There should be $N_\text{code}$ outputs taking values $\vec{y} = (y_j)_{j=0,...,N_\text{code}}.$

Normalise all the rewards so that for each bit $j,$ $\rt{j} + \left(N_\text{code} - 1\right)\rf{j} = 0.$  In other words
\begin{equation}
	r_{jk}
	\gets
	r_{jk} - \frac{ \rt{j} + \left(N_\text{code} - 1\right)\rf{j}}{N_\text{code}}
	.
\end{equation}

The $Q$ estimate is then taken to be
\begin{equation}
	Q(\vec{x})
	=
	\sum_j b_j y_j
	\equiv
	\sum_j \abs{y_j}
	,		
\end{equation}
where
\begin{equation}
	b_j = \operatorname{sgn} (y_j)
\end{equation}
is the prediction for the machine value of the $j$th bit.  The loss function is
\begin{equation}
	L
	=
	\abs{Q(\vec{x}) -r}^2
	. 
\end{equation}

\subsection{Bob}

Bob receives a matrix, $\mat{X} = (\vec{X}_{i}) = (X_{ij})$ for $0\leq i < N_\text{select},\ 0\leq j < N_\text{code},$ and a code $\vec{c}=(c_j)_{j=0,...,N_\text{code}}.$  Why not makes his outputs be $Q$-estimates $\vec{z} = (z_i)_{i=0,...,N_\text{select}}.$ Bob's prediction is then
$
	\vec{x}_\text{pred}
=
\vec{X}_{i_\text{pred}}
$
where
\begin{equation}
	i_\text{pred}
	=
	\operatorname{argmax}_{i}  (z_i)
	.
\end{equation}
The loss function is 
\begin{equation}	\label{eq:Bob_loss_fn}
	L
	=
	\abs{\vec{z}_{i_\text{pred}} -r}^2
	. 
\end{equation}

How do we enforce covariance with respect to the order of $(\vec{X}_{i})$?
\begin{enumerate}
	\item Covariance will occur naturally and quickly without any specific intervention.  {\em To be determined.}
	\item Covariance can be enforced through choosing a set $\left\lbrace \sigma \right\rbrace \subseteq S_{n_\text{code}},$ which could be generated element--by--element by composing randomly--selected basis transpositions $(j\ j+1),$ and then adding to the loss a term
	\begin{equation}
		\mu\sum_\sigma\abs{\vec{z} - \sigma^{-1}\left[\vec{z}(\sigma[\mat{X}])\right]}^2
	\end{equation}
	for some fixed hyperparameter $\mu > 0.$  Note this the term is still run backward through the original $\vec{x}\mapsto\vec{z}$ net configuration only.  {\em How effective would that be?  How big does $\left\lbrace \sigma \right\rbrace$ have to be? And how much time would the permutation and the additions forward passes cost?}
	\item Enforce covariance via direct identification of weights in Bob's net.  {\em How?}
	\item Something related to set transformers. {\em ?}
	\item Adopt a different basic set--up where each $\vec(X)_i$ is fed through the net separately, alongside the code $\vec{c},$ resulting in a $Q$-estimate $\vec{z}_i.$  Then find the loss function as in \eref{eq:Bob_loss_fn}.  {\em Seems the most straightforward?} {\bf Adopt this for now.}
\end{enumerate}
None of these quite amount to Bob seeks to reproduce the Alice's code vocabulary.  However Bob could additionally set up a net in the same basic configuration as Alice's (he doesn't know the weights of course) and train {\em that} net jointly with his main net.

%%%%%%%%%%%%%%%%%%%%%%%%%%%%%%%%%%%%
%%%%%%%%%%%%%%%%%%%%%%%%%%%%%%%%%%%%
%%%%%%%%%%%%%%%%%%%%%%%%%%%%%%%%%%%%

\bibliographystyle{JHEP}
%\bibliography{guesstuples_project}

\end{document}

